\section{Conclusions}
We have presented a simulation-based framework for quantifying the impact of optical ghosts on LSSTCam imaging, motivated by the need to assess compliance with the LSST system requirements.We combine the Yale Bright Star Catalog, empirically calibrated magnitude transformations, and optical ray-tracing simulations using \texttt{Batoid}, to estimate both the morphology and surface brightness of the dominant ghosts and measure the resulting ghost-impacted area on the focal plane.

Applying this framework to $\sim$3100 visits from the w37 intermittent DRP, we find that the average fraction of the LSSTCam focal plane significantly impacted by ghosts is $0.57\%$ when averaged across all bands. This value is below the 1\% threshold specified by the LSST system requirement when interpreted in an ensemble-averaged sense. The impacted area exhibits strong band dependence, with the largest contributions occurring in the $u$ band due to its lower sky background, and minimal impact in the $r$ and $i$ bands. It The impacted area is also heavily dependent on the field being observed, as ndividual visits containing very bright stars can show substantially larger impacted fractions.

We further explored the dependence of ghost-impacted area on stellar magnitude and off-axis position, finding that star brightness is the dominant driver of ghost impact, with only a weak dependence on field angle. Using these results, we constructed all-sky maps of expected ghost-impacted area based on the Yale Bright Star Catalog.

Overall, our results demonstrate that optical ghosts are a non-negligible source of contamination for low-surface-brightness science in LSST, particularly in the presence of very bright stars. 