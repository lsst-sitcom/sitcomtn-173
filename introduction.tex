\section{Introduction}
Optical ghosts are spurious reflections that appear in optical imaging systems as a result of multiple reflections of light off the optical elements in the system. Although ghosts are created by all astronomical sources, most of them are undetectable and have a negligible impact on the image. However, ghosts produced by bright stars can be quite impactful even if the reflectances of the optical elements are small (${\sim}1\%$) because the total number of photons from these stars are quite large. Figure~\ref{fig:optical_ghosts} shows an example of optical ghosts created by a bright star in the top right corner of the LSST Camera focal plane during commissioning of the Vera C.\ Rubin Observatory. %2025110500406. 

\begin{figure}[h]
    \centering
    \includegraphics[width=0.5\textwidth]{figures/lsstcam_focal_plane_mosaic_2025-11-05_000406.jpg}
    \caption{Post-ISR image of the LSSTCam focal plane (visit=2025110500406; band=$i$) with optical ghosts created by a bright star in the top right corner. } 
    \label{fig:optical_ghosts}
\end{figure}

Ghosts produced by bright stars often pose challenges to sky background estimation and photometric measurements. They are also sources of contamination for low-surface-brightness science. Measuring the area of the focal plane impacted by these artifacts is therefore crucial to assess the impact on these science cases.

The LSST System Requirement\footnote{OSS Requirements: \url{https://docushare.lsst.org/docushare/dsweb/Get/LSE-30}; LSR Requirements: \url{https://docushare.lsst.org/docushare/dsweb/Get/LSE-29}} on the impact of optical ghosts states that the: ``Percentage of image area that can have ghosts with surface brightness gradient amplitude of more than $1/3$ of the sky noise over 1 arcsec shall be less than 1\%''. 
The wording of these requirements was intentionally left vague to allow subsequent groups to define it as appropriate. We opt to define it as the total area of the focal plane containing a ghost with a flux-to-sky-noise ratio exceeding a value of 1/3 on any detector. In order to satisfy this requirement, this value should not exceed 1\%. We also assume that this requirement applies to the ensemble of visits, as opposed to every individual visit.

Measuring the fluxes of the ghosts with on-sky data to estimate the ghost flux-to-sky-noise ratio is challenging. Uneven flat-fielding during commissioning led to gradients in the mean sky counts across the focal plane, changing the baseline of the brightness of the ghosts from one side to the other. Most of the prominent ghosts were also found to be spatially degenerate, with some ghosts fully overlapping others. This made fitting the amplitude of each individual ghost difficult. To circumvent these challenges, we chose instead to calculate the expected impact of ghosts using simulations that were calibrated to commissioning data.